\documentclass{article}

\usepackage[margin=1.5in]{geometry} % Please keep the margins at 1.5 so that there is space for grader comments.
\usepackage{amsmath,amsthm,amssymb,hyperref}

\newcommand{\R}{\mathbf{R}}  
\newcommand{\Z}{\mathbf{Z}}
\newcommand{\N}{\mathbf{N}}
\newcommand{\Q}{\mathbf{Q}}

\newenvironment{theorem}[2][Theorem]{\begin{trivlist}
\item[\hskip \labelsep {\bfseries #1}\hskip \labelsep {\bfseries #2.}]}{\end{trivlist}}
\newenvironment{lemma}[2][Lemma]{\begin{trivlist}
\item[\hskip \labelsep {\bfseries #1}\hskip \labelsep {\bfseries #2.}]}{\end{trivlist}}
\newenvironment{claim}[2][Claim]{\begin{trivlist}
\item[\hskip \labelsep {\bfseries #1}\hskip \labelsep {\bfseries #2.}]}{\end{trivlist}}
\newenvironment{problem}[2][Problem]{\begin{trivlist}
\item[\hskip \labelsep {\bfseries #1}\hskip \labelsep {\bfseries #2.}]}{\end{trivlist}}
\newenvironment{proposition}[2][Proposition]{\begin{trivlist}
\item[\hskip \labelsep {\bfseries #1}\hskip \labelsep {\bfseries #2.}]}{\end{trivlist}}
\newenvironment{corollary}[2][Corollary]{\begin{trivlist}
\item[\hskip \labelsep {\bfseries #1}\hskip \labelsep {\bfseries #2.}]}{\end{trivlist}}

\newenvironment{solution}{\begin{proof}[Solution]}{\end{proof}}

\begin{document}

\large % please keep the text at this size for ease of reading.

% ------------------------------------------ %
%                 START HERE             %
% ------------------------------------------ %

{\Large Session 8 % Replace with appropriate page number 
\hfill  Advanced Linear Programming I}

\begin{center}
{\Large Parham Alvani - 98131910} % Replace "Author's Name" with your name
\end{center}
\vspace{0.05in}

% -----------------------------------------------------
% The "enumerate" environment allows for automatic problem numbering.
% To make the number for the next problem, type " \item ". 
% To make sub-problems such as (a), (b), etc., use an "enumerate" within an "enumerate."
% -----------------------------------------------------

\begin{enumerate}

\item 3 Review Problems Problem 51
\begin{itemize}
\item \(x_i\) number of transistor that is created from method \(i\)
\item \(y_j\) number of transistor in the refine process from grade \(j\)
\end{itemize}
\par
Objective Function:
\[
    \min\quad 50 x_1 + 70 x_2 + 25 (\sum_{j=0}^3y_j)
\]
s.t.
\[
    y_0 \le 0.3 x_1 + 0.2 x_2
\]
\[
    y_1 \le 0.3 x_1 + 0.2 x_2
\]
\[
    y_2 \le 0.2 x_1 + 0.25 x_2
\]
\[
    y_3 \le 0.15 x_1 + 0.20 x_2
\]
\[
    x_1 + x_2 + y_0 + y_1 + y_2 + y_3 \le 20000
\]
\[
    0.3 x_1 + 0.2 x_2 - y_1 + 0.25 y_0 + 0.3 * y_1 \ge 3000
\]
\[
    0.2 x_1 + 0.25 x_2 - y_2 + 0.15 y_0 + 0.3 y_1 + 0.4 y_2 \ge 3000
\]
\[
    0.15 x_1 + 0.2 x_2 - y_3 + 0.2 y_0 + 0.2 y_1 + 0.3 y_2 + 0.5 y_3 \ge 2000
\]
\[
    0.05 x_1 + 0.15 x_2 + 0.1 y_0 + 0.2 y_1 + 0.3 y_2 + 0.5 y_3 \ge 1000
\]
\[
    0 \le x_i \quad \forall i \in {1,2}
\]
\[
    0 \le y_j \quad \forall j \in {0,1,2,3}
\]

\end{enumerate}

% ---------------------------------------------------
% Anything after the \end{document} will be ignored by the typesetting.
% ----------------------------------------------------

\end{document}